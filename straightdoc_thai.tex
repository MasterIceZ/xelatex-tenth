\documentclass[10pt,twoside,thai,sansthai,notypist]{straightdoc}

\usepackage[usenames,dvipsnames,svgnames,table]{xcolor}
\newrobustcmd{\codehl}[1]{\textlatin{\texttt{\color{RoyalBlue} #1}}}

\title{คลาส \texttt{straightdoc}}
\author{อาภาพงศ์ จันทร์ทอง}
\date{24 เมษายน 2017}

\begin{document}
    \maketitle

    นี่คือเอกสารตัวอย่างที่ใช้คลาส \codehl{straightdoc} เพื่อเขียนเอกสารภาษาไทย โดยกำหนด \codehl{option} ชื่อ \codehl{thai} ให้กับคลาสของเอกสารนี้ \textlatin{And it could be used in conjunction with Latin script, as you can see here. This par of the sentence is using Latin font,} whereas this part is using the Thai font.

    แม้ว่าตอนนี้ฟอนต์ภาษาไทยอาจจะดูใหญ่เกินไปบ้าง โปรดรออัปเดตถัดไปนะ

    หากไม่กำหนด \codehl{option} ชื่อ \codehl{notypist} แล้วเอกสารนี้จะใช้ฟอนต์ Thai Typist ไปโดยปริยายสำหรับ Teletyped Font ซึ่งอาจดูไม่สวยเท่า Inconsolata เป็นต้น

    การกำหนด \codehl{option} ชื่อ \codehl{sansthai} ทำให้เอกสารนี้ใช้ฟอนต์สำหรับภาษาไทยเป็นแบบ Sans Serif เป็นค่าปริยาย แต่จะไม่ส่งผลต่อค่าปริยายของฟอนต์สำหรับภาษาอังกฤษ

\end{document}
