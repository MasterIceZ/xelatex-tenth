\usepackage{lipsum}
\title{คลาส \texttt{tenth}}
\author{อาภาพงศ์ จันทร์ทอง}

\newrobustcmd{\codehl}{\texttt}

\begin{document}
    \maketitle

    \section{บทนำ \label{sec:intro}}
    \subsection{บางอย่าง}
    \subsubsection{บางอย่างอีก \label{ssub:wow}}

    นี่คือเอกสารตัวอย่างที่ใช้คลาส \codehl{tenth} เพื่อเขียนเอกสารภาษาไทย โดยกำหนด \codehl{option} ชื่อ \codehl{thai} ให้กับคลาสของเอกสารนี้ \texteng{And it could be used in conjunction with Latin script, as you can see here. This par of the sentence is using Latin font,} whereas this part is using the Thai font.

    \begin{eng}
        English text here. {\tha และนี่ภาษาไทย}
    \end{eng}

    \autoref{sec:intro} a \sectionautorefname{} b \autoref{eq:a} \autoref{ssub:wow} \autoref{item:j}
    \Hfootnoteautorefname{} \AMSautorefname{} \autoref{thm:asdfdasf} \autoref{lem:ad} \lemmaautorefname{}

    {\selectlanguage{english} This is english \autoref{thm:asdfdasf} and \chaptername}

    \begin{theorem}
        \label{thm:asdfdasf}
        นะจ๊ะ
    \end{theorem}
    \unskip
    \begin{lemma}
        \label{lem:ad}
        นะจ๊ะ
    \end{lemma}

    \begin{remark}
        นะจ๊ะ
    \end{remark}

    แม้ว่าตอนนี้ฟอนต์ภาษาไทยอาจจะดูใหญ่เกินไปบ้าง โปรดรออัปเดตถัดไปนะ

    หากไม่กำหนด \codehl{option} ชื่อ \codehl{notypist} แล้วเอกสารนี้จะใช้ฟอนต์ Thai Typist ไปโดยปริยายสำหรับ Teletyped Font ซึ่งอาจดูไม่สวยเท่า Inconsolata เป็นต้น

    การกำหนด \codehl{option} ชื่อ \codehl{sansthai} ทำให้เอกสารนี้ใช้ฟอนต์สำหรับภาษาไทยเป็นแบบ Sans Serif เป็นค่าปริยาย แต่จะไม่ส่งผลต่อค่าปริยายของฟอนต์สำหรับภาษาอังกฤษ

    \begin{enumerate}
        \item อพาร์ทเมนต์รีทัชสเปกเคส ปาสคาลเที่ยงวัน เรตติ้งจึ๊กเนิร์สเซอรี่ ไนท์มัฟฟินช็อปปิ้ง จ๊อกกี้น็อค โปสเตอร์ออดิทอเรียมว้อยฟีด แชมพูทิปอินดอร์ไทม์ พาสเจอร์ไรส์ศิลปวัฒนธรรมทับซ้อนต้าอ่วย วิลล์ คันยิแลนด์นายพราน เพียวพล็อตซาตาน โบตั๋น คอลัมนิสต์มลภาวะโพลารอยด์ลาเต้แรงผลัก คาปูชิโนเซลส์แมนแลนด์โรแมนติก แรงใจเป็นไงเต๊ะเฟอร์นิเจอร์โอยัวะ ฮิปโปราเมนดิสเครดิต

            นิวทัวริสต์เคลมคอร์ป คอลเล็กชั่นโปรเจกเตอร์ต่อรองหยวนพลาซ่า ทรู เทรลเลอร์ สกายคาปูชิโนเทวาธิราช เฮอร์ริเคนฮาลาลแฮนด์ ภควัทคีตาฮัมวโรกาสวอลซ์ ออร์แกนหงวน ซิ้มสโรชา ลิสต์ อิ่มแปร้แรงดูดลาตินแจ็กพอตดีลเลอร์ เจไดปฏิสัมพันธ์เพาเวอร์ วอล์กเฉิ่ม แจ๊กพ็อตสโรชาคาร์ ลาตินแฟ็กซ์ซีนเอ๋กรีน พุดดิ้งพาร์ซูม

        \begin{enumerate}
            \item 1
            \item 2
        \end{enumerate}

        \item  \label{item:j} ธรรมาเดบิตซูชิพุดดิ้งโปรเจกต์ วีซ่า อินเตอร์นินจานินจา ไพลิน ออร์เดอร์ฮาร์ด ทัวร์นาเมนท์ไอเดียโลโก้กฤษณ์ นรีแพทย์คอลเล็กชั่นความหมายวอล์ก บอดี้บาบูนคาวบอย โซลาร์ชิฟฟอนอึ้มคอนโดรีเสิร์ช เนิร์สเซอรี่แอปเปิลตรวจทานสต๊อคแชมเปี้ยน บาร์บีคิวฮากกาเรตติ้งซาตานเอาท์ เซรามิกรีสอร์ท บอดี้ต่อยอดบอดี้รองรับไวกิ้ง ไฮเปอร์วโรกาสภารตะ พาเหรดแฟรนไชส์เทคโน เป็นไง

            พุดดิ้งไฮเอนด์ฟยอร์ดแอปเปิ้ล แรงใจสเตชันเกสต์เฮาส์ เฟรชแซลมอน ผ้าห่มเปปเปอร์มินต์ม้งโมเดิร์น เฟรชมายาคติ สไปเดอร์ซิ้มฮอตฟีดปฏิสัมพันธ์ วโรกาส ไฮแจ็คฮีโร่โทรโอยัวะไฟลต์ สป็อตอีโรติกเทเลกราฟ คอรัปชัน ไหร่บ๋อยอวอร์ดศิลปากรเทรด อัตลักษณ์ซากุระโปรเจ็ค ลีเมอร์ธรรมาซาบะแต๋วจ๊อกกี้ แมมโบ้ จอหงวนซะแยมโรลวิปเอสเพรสโซ มิลค์

            ทอล์ค เซี้ยวช็อตมุมมอง อัลไซเมอร์คอร์รัปชั่นหมั่นโถว เบญจมบพิตร คาวบอย ซาบะ คอนแทครวมมิตร แพทยสภาแฟรี ป๋อหลอทอล์คโซลาร์คันยิ สตรอเบอร์รีช็อปปิ้งโรลออน สเตริโออัลมอนด์ทับซ้อนอิ่มแปร้หมั่นโถว สตาร์ทเอาต์กรรมาชนแจ๊กพ็อต ฟรังก์คำสาปบาบูนฮาลาลทอม เสือโคร่งซิมโฟนี่ เพนตากอนแหม็บแดนซ์คอร์รัปชัน เกมส์
    \end{enumerate}

    \label{page:y}
    \begin{equation}
        1 + 1 = 2 \label{eq:a}
    \end{equation}

    \section{Introduction}
    \label{sec:intro}

    \lipsum[1]

\begin{lstlisting}[language=python,caption={A square function, obviously.}]
def square(x):
    return x ** 2
\end{lstlisting}

    \marginnote{This is a sidenote in the margin, useful for uninterrupting note.}
    \lipsum[2]

    There are 2--4 people around this house at 3--5 \textsc{am}. \marginnote{\emph{Actually, they are here to shout} $E = mc^2$.} Do not worry; it is totally normal. These are \colorsq{red}, \colorsq{green}, and \colorsq{blue}\ respectively.

    Here is an example.

    \begin{equation}
        x^2 + y^2 = z^2  \label{eq:pythagorean}
    \end{equation}

    This is an example of \eqref{eq:pythagorean}. \lipsum[3]

    \begin{example}
        This is an example of \eqref{eq:pythagorean}.
    \end{example}

    \lipsum[4]

    \begin{example*}[likeme]
        This is an example of \eqref{eq:pythagorean}.
    \end{example*}

    \lipsum[5]
    \begin{note}
        This is an example of \eqref{eq:pythagorean}. \rsq\gsq\bsq\ysq -- {-}{-}
    \end{note}
    \lipsum[6]

    \begin{itemize}
        \item  First item
        \item  Second item
            \begin{itemize}
                \item  First subitem
                    \begin{itemize}
                        \item  What the hell!
                    \end{itemize}
                \item  Second subitem
            \end{itemize}
    \end{itemize}


\begin{lstlisting}[language=pseudocode,escapeinside={<}{>}]
one < \SuppressNumber >
two
three < \ReactivateNumber >
four < \SuppressNumber >
five < \ActivateNumber{7} >
six
seven < \SuppressNumber\ActivateNumber{13} >
eight
if then while true
\end{lstlisting}

    \begin{enumerate}
        \item \lipsum[1-3]
        \item \lipsum[2-4]
            \begin{enumerate}
                \item \lipsum[3-5]
                    \begin{enumerate}
                        \item \lipsum[3-5]
                        \item \lipsum[4-6]
                    \end{enumerate}
                \item \lipsum[4-6]
            \end{enumerate}
        \item \lipsum[2-4]
    \end{enumerate}

    \begin{description}
        \item[One time you are here] \lipsum[1-2]
        \item[Second item] \lipsum[3-4]
    \end{description}

\end{document}
